\documentclass[a4paper,10pt]{article}
\usepackage[french]{babel}
\usepackage[utf8]{inputenc}
\usepackage[left=2.5cm,top=2cm,right=2.5cm,nohead,nofoot]{geometry}
\usepackage{url}
\usepackage{graphicx}
\usepackage{hyperref}

\linespread{1.1}



\begin{document}

\begin{titlepage}
\begin{center}
\textbf{\textsc{UNIVERSIT\'E LIBRE DE BRUXELLES}}\\
%\textbf{\textsc{Faculté des Sciences}}\\
%\textbf{\textsc{Département d'Informatique}}
\vfill{}\vfill{}
\begin{center}{\Huge Rapport : HORECA}\end{center}{\Huge \par}
\begin{center}{\large Thomas Perale}\end{center}{\Huge \par}
\vfill{}\vfill{} \vfill{}
\begin{flushleft}{\large \textbf{INFO-H-303 Base de données}}\hfill{Esteban Zimányi, Michaël Waumans}\end{flushleft}{\large\par}
\vfill{}\vfill{}\enlargethispage{3cm}
\textbf{Année académique 2015-2016}
\end{center}
\end{titlepage}

%\begin{abstract}
%Ce rapport présente ...
%\end{abstract}


\tableofcontents

\pagebreak


\section{Diagramme entité association}
\subsection{Diagramme}
\begin{figure}[hbt]
  \includegraphics[scale=0.4]{bdd.png}
  \caption{Diagramme entité association
}
\end{figure}
\subsection{Contraintes}
Les contraintes sont les suivantes :
\begin{itemize}
  \item Le couple (Longitude, Latitude) est unique.
  \item La longitude et la latitude doivent être comris entre -180 et 180.
  \item Le nom d'utilisateur est unique.
  \item L'email d'utilisateur est unique.
  \item Pour les commentaire, le couple (id_utilisateur, date) est unique.
  \item Le nombre d'étoile (commentaire et hotel), doit être compris entre 1 et
      5.
  \item Pour une Station, le nombre de vélo dont la dernière Location finit  dans cette station ne doit pas dépasser sa capacité,
  \item La Date de création d'un établissement doit précéder celle de ses
      commentaires.
  \item La Date d'enregistrement d'un admin doit précéder celle de création
      d'un établissement par celui-ci.
  \item Un même utilisateur ne peut pas écrire deux commentaire en même temps.
\end{itemize}



\section{Modèle relationnel}
\subsection{Modèle}


\begin{description}
\item[] \textbf{Utilisateur}(\underline{Id}, Email, MotDePasse, DateEnregistrement, Admin(0, 1))
    \begin{description}
        \item[] Utilisateur.Admin indique si l'utilisateur est un admin ou non.
    \end{description}

\item[] \textbf{Etablissement}(\underline{Id}, Nom, Adresse.Rue,
    Adresse.Numéro, Adresse.CodePostal, Adresse.Localité, Coordonnée.Latitude,
    Coordonnée.Longitude, Téléphone, Site, DateCreation, Creator)
    \begin{description}
        \item[] Etablissement.Creator réfèrence Utilisateur.Id.
    \end{description}



\item[] \textbf{Restaurant}(\underline{Id}, Prix, Places, AEmporter, Livraison, horaire)
    \begin{description}
        \item[] Restaurant.Id réfèrence Etablissement.Id.
    \end{description}

\item[] \textbf{Bar}(\underline{Id}, Fumeur, Snacks)
    \begin{description}
        \item[] Bar.Id réfèrence Etablissement.Id.
    \end{description}

\item[] \textbf{Hotel}(\underline{Id}, NombreEtoile, NombreChambre, Prix)
    \begin{description}
        \item[] Hotel.Id réfèrence Etablissement.Id.
    \end{description}

\item[] \textbf{Commentaire}(\underline{Id}, Titre, Commentaire, NombreEtoile, Date,
    Image, IdArticle, IdUtilisateur)
    \begin{description}
        \item[] Commentaire.IdArticle réfèrence Etablissement.Id.
        \item[] Commentaire.IdUtilisateur réfèrence Utilisateur.Id.
    \end{description}

\item[] \textbf{Label}(\underline{Id}, Label, IdArticle, IdUtilisateur)
    \begin{description}
        \item[] Commentaire.IdArticle réfèrence Etablissement.Id.
        \item[] Commentaire.IdUtilisateur réfèrence Utilisateur.Id.
    \end{description}
\end{description}


\section{Hypothèses}

Les utilisateur "admin" sont encodé directement dans la base de donnée.

\section{Justification}

\end{document}
